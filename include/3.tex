% !TEX program = lualatex
% !TEX root = ../sets.tex
% !TEX spellcheck = en_GB

\section{Empty, pair, union and power set}

We already now there is the empty class (see construction~\ref{construction:EmptyClass}), but we actually want it to be a set.

\begin{axiom}[Empty set]
\(\nil\) is a set.
\end{axiom}

Now, we have the empty set, but we also want sets that are not empty.

\begin{axiom}[Pair set]\label{axiom:PairSet}
For every sets \(A\) and \(B\) the class \(\set{A, B}\) is a set.
\end{axiom}

A {\em singleton} is a class which has a unique element. The previous axiom also implies singletons are sets too:  if \(x\) is a set, so \(\set{x, x}\) is; usually one avoids repetitions and simply writes \(\set{x}\).

%\begin{definition}
%For \(X\) class, we have the class
%\[\bigcup X \coloneq \set{A \mid \exists B \in X : A \in B}.\]
%Sometimes, instead of \(\bigcup X\), we write \(\bigcup_{E \in X} E\).
%\end{definition}

\begin{axiom}[Union set]\label{axiom:UnionSet}
For every set \(X\) the class %\(\bigcup X\)
\[\bigcup X \coloneq \set{A \mid \exists B \in X : A \in B}\]
is a set.
\end{axiom}

Sometimes, instead of \(\bigcup X\), we write \(\bigcup_{E \in X} E\).

\begin{theorem}
For every sets \(A\) and \(B\) the class \(A \cup B\) is a set.
\end{theorem}

\begin{proof}
If \(A\) and \(B\) are sets, so \(\set{A, B}\) is by theorem~\ref{axiom:PairSet}. Finally, because of theorem~\ref{axiom:UnionSet}, \(A \cup B = \bigcup \set{A, B}\) is a set.
\end{proof}

\begin{axiom}[Power set]\label{axiom:PowerSet}
For every set \(X\) the class
\[\wp X \coloneq \set{E \mid E \subseteq X}\]
is a set, which we call {\em power set}.
\end{axiom}